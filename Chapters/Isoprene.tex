% CHAPTER 3 (probably)
% Isoprene Emissions

\chapter{Isoprene Emissions in Australia} % Main chapter title
\label{ch_isop}

\section{GEOS-Chem isoprene mechanisms}
\label{ch_isop:sec:GEOSChemMechanisms}
  \subsection{Outline}
    The isoprene reactions simulated by GEOS-Chem were originally based on \cite{Horowitz1998}.
    This involved simulating NO$_X$, O$_3$, and NMHC chemistry in the troposphere at continental scale in three dimensions, with detailed NMHC chemistry with isoprene reactions and products.
    The mechanism was subsequently updated by \citet{Mao2013}, who change the isoprene nitrates yields and add products based on current understanding as laid out in \citet{Paulot2009a,Paulot2009b}.
    Further mechanistic properties, like isomerisation rates, are based on results from four publications: cite{Crounse2011,Crounse2012,Peeters2010,Peeters2011}.
    (TODO: check abstracts Peeters papers).
    \cite{Crounse2011} examines the isomerisations associated with the oxidation of isoprene to six different isomers (ISO$_2$) formed in the presence of oxygen: isoprene $ + OH \rightarrow^{O_2} $ ISO$_2$.
    They determine rates and uncertainties involved in these reactions, and study the rate of formation of C$_5$-hydroperoxyaldehydes (HPALDs) by isomerisation.
    \cite{Crounse2012} examine the fate of methacrolein (MACR), one of the products of isoprene oxidation. 
    Prior to this work MACR oxidation chamber studies were performed in high NO or HO$_2$ concentrations, giving peroxy lifetimes of less than 0.1~s.
    In most environments this is not the case, and MACR products over various NO concentrations and peroxy radical lifetimes are determined in their work.
    \cite{Peeters2010} examine photolysis of hydroperoxy-methyl-butenals (HPALDs, produced by isoprene isomerisation), which regenerates OH levels in areas with high isoprene emissions.
    Additionally, photolysis of photolabile peroxy-acid-aldehydes creates OH and improved model aggreement with continental observations.
   The OH and HPALD interactions are central to maintaining the OH levels in pristine and moderately polluted environments, which makes isoprene both a source and a sink of OH TODO: cite and DL;\url{http://www.nature.com/ngeo/journal/v5/n3/full/ngeo1405.html}.
    
    Formation of isoprene nitrates have an effect on ozone levels through NO$_X$ sequestration, and the yields and destinies of these nitrates is analysed in \citet{Paulot2009a}. 
    They use anion chemical ionization mass spectrometry (CIMS) to determine products of isoprene photooxidation.
    In a chamber with clean air and high NO concentrations, isoprene photooxidation is initially driven by OH addition, followed by NO$_X$ chemistry (150~min - 600~min), and finally HO$_X$ dominated chemistry.
    The yields of various positional isomers of isoprene nitrates is estimated, and pathways of their oxidation products is shown and used in the GEOS-Chem isoprene mechanism \citep{Paulot2009a,Mao2013}. 
    
    In low NO$_X$ conditions, isoprene oxidises to yield 70\% hydroxyhydroperoxides (ISOPOOH), which then oxidises to create dihydroxyperoxides (IEPOX) with OH recycling maintaining the OH levels in the atmosphere \citep{Paulot2009b}.
    In older models isoprene produced ISOPOOH which then titrated OH, however, the loss of OH has not been seen in measurements \citep{Paulot2009b,Mao2013}.
    The isoprene mechanism in GEOS-Chem has been updated to include OH regeneration from oxidation of epoxydiols and slow isomerisation of ISOPO$_2$ \citep{Mao2013}.
    
    Under high NO$_X$ conditions, isoprene undergoes OH addition at the 1 and 4 positions, becoming $\beta$ (71\%) or $\delta$ (29\%) hydroxyl peroxy radicals (ISOPO$_2$). 
    The $\beta$-hydroxyl reacts with NO$_X$ and produces HCHO (66\%), methylvinylketone (40\%) (MVK), methacrolein (26\%), and $\beta$-hydroxyl nitrates (6.7\%) (ISOPNB).
    The $\delta$-hydroxyl reacts with NO to form $\delta$-hydroxyl nitrates (24\%) (ISOPND), and ISOPNB (6.7\%).
    ISOPNB and ISOPND yield first generation isoprene at 4.7\% and 7\% respectively.
    
    Under low NO$_X$ conditions, ISOPO$_2$ may react with HO$_2$ to form ISOPOOH.
    In this case there is also production of HCHO (4.7\%), MVK(7.3\%), and MACR (12\%).
    As stated in earlier; most ISOPOOH will form IEPOX (epoxydiols) after reacting with OH and lead to OH regeneration.
    The other mechanism in low NO$_X$ environments is unimolecular isomerisation of ISOPO$_2$.
    This leads to production of hydroperoxyaldehydes (HPALDS), which generally photolyse and have an OH yield of 100\%.
    \citet{Mao2013} show that a lower (factor of 50) rate constant for ISOPO$_2$ isomerisation leads to better organic nitrate aggreements with ICARTT. 
    
    This update leads to more accurate modelling of OH concentrations, especially in low NO$_X$ conditions common in remote forests.
    Prior to \citet{Mao2012}, measurements of OH in high VOC regions may have been up to double the real atmospheric OH levels, due to formation of OH inside the instrument.
    \citet{Mao2012} examine an upgraded method of measurement, and compare these against a regional atmospheric chemistry model (RACM2), with the OH recycling updates from \citet{Paulot2009b} as discussed in prior paragraphs.
    
    The updates to isoprene chemistry by \citet{Mao2013}, and those shown in \cite{Crounse2011,Crounse2012} are the last before version 11, which was not used in this work.

    The full current mechanism is described online at \url{http://wiki.seas.harvard.edu/geos-chem/index.php/New_isoprene_scheme}.
    
  \subsection{Emissions from MEGAN}
    MEGAN simulates biogenic emissions of various gases including isoprene, based on various meteorological, land cover, and plant type parameterisations.
    
    One of the important parameters in Australia is the soil moisture activity factor($\gamma_{SM}$), which can have large regional affects on the isoprene emissions \citep{Sindelarova2014,Bauwens2016}.
    Generally if soil moisture is too low, isoprene emissions stop \citep{Pegoraro2004,Niinemets2010}, however in many Australian regions the plants may be more adapted to lower moisture levels. (TODO: Find cites for this - talk from K Emerson at Stanley indicated this)
    GEOS-Chem runs MEGANv2.1, which has three possible states for isoprene emissions based on the soil moisture ($\theta$):
    \begin{align*}
      \gamma_\mathrm{SM} & = 1 && \theta > \theta_1 \\
      \gamma_\mathrm{SM} & = (\theta-\theta_w)/\Delta\theta_1  && \theta_w < \theta < \theta_1 \\
      \gamma_\mathrm{SM} & = 0 && \theta < \theta_w \\
    \end{align*}
    where $\theta_w$ is the wilting point, and $\theta_1$ determines when plants are near the wilting point.
    The wilting point is set by a land based database from \citet{Chen2001}, while $\theta_1$ is set globally based on \citet{Pegoraro2004}.
    
    In GEOS-Chem the emissionscan be globally multiplied by a constant factor, which was performed to determine the smearing and sensitivity.
    By running the model two extra times, with the biogenic emissions set to zero and one half, while other parameters remain unchanged, the general affects of isoprene emissions which the model undergoes can be determined.
    
%----------------------------------------------------------------------------------------
%	SECTION
%----------------------------------------------------------------------------------------
\section{Isoprene emissions estimation}
\label{ch_isop:sec:IsopreneEmissions}

  \subsection{Outline}
    With the vertical columns of biogenic HCHO we can infer the local (grid space) isoprene emissions using effective molar formaldehyde yield (In other continents around 2-3, or 1 in low NO$_X$ conditions) \citep{Palmer2003,Marais2012,Bauwens2016}.
    If we assume there is fast HCHO yield, so that the effect of chemical transport is minimal, and that HCHO and isoprene are at steady states, then we can calculate local yield from our CTM.
    %This yield is derived from both HCHO and isoprene, such as was used by \citet{Millet2006} who produced a molar HCHO yield of 2.3 in north eastern USA.
    Yield is calculated from the modelled slope between isoprene emissions and HCHO total column within each gridbox over Australia, as performed in \cite{Palmer2003}, using modelled values between 1300-1400 LT which is around the overpass time of the OMI.
    This modelled yield is then used in conjunction with the recalculated OMI measurements in order to estimate isoprene emissions.
    
    The calculations used to determine isoprene emissions over Australia are fully described in \ref{ch_isop:sec:EmissionCalculation} and follow the method of \citet{Palmer2003}.
    To calculate emissions we use a reduced major axis (RMA) regression between modelled average (from 1300-1400 LT) values of the loss rates and total columns, an example is shown in figure TODO: figure with RMA of these over whatever time and space I end up using.
    
    The measured background HCHO is the average concentration measured in the remote pacific at the same time.
    The modelled background is determined from a run with isoprene emissions turned off, which allows us to see exactly how much the modelled isoprene emissions alter each vertical column of HCHO.
    
    Isoprene quickly forms HCHO in the atmosphere when in the presence of high levels of NO$_X$.
    However, over Australia NO$_X$ levels are generally not high enough and we must take extra care that we can account for the transport or 'smearing' caused by slower HCHO formation.
    Smearing sensitive grid boxes within the model can be detected by running the model with two uniformly differing isoprene emission levels, then finding the grid boxes where the changed HCHO column is greater than can be attributed to local emission difference.
    Using equation \ref{ch_isop:eqn:isop_yield} with two different isoprene emission levels:
    \begin{equation*}
      \hat{S} = \frac{\Delta~\Omega_{HCHO}}{\Delta~E_{ISOP}}
    \end{equation*}
    Consider halving the isoprene emitted globally and rerunning the model, if the local grid HCHO is reduced by much more than half (factoring yield) then you can infer sensitivity to non-local isoprene emissions.
    This can be dependent on local or regional weather patterns, as greater wind speeds will reduce the time any emitted compound stays within the local grid box.
    As such smearing sensitivity is both spatially and temporally diverse, shown in figure TODO: is a picture of the smearing sensitivity over Australia.
   
    Once the smearing sensitive grid squares are filtered out, application of equation \ref{ch_isop:eqn:isop_yield} gives us an estimate of isoprene emissions across the nation.
    
    Most recently a \citet{Bauwens2016} undertook a similar process to what I am doing, although with slightly different focus, using the IMAGESv2 global CTM instead of GEOS-Chem.
    They calculate emissions which create the closest match between model and satellite vertical columns, and compare these postiori data with the apriori (satellite data) and independent data sets.
    (TODO: simple outline of what they did and how my focus is different, this paper will also need to be summarised in the LitReview)

  \subsection{HCHO Products and yield}
    % Yields and Tables moved to BiogenicIsoprene chapter
    
  \subsection{CAABA/MECCA yield}
    % Moved to Biogenic Isoprene Chapter
  
  \subsection{Calculation of Emissions}
    \label{ch_isop:sec:EmissionCalculation}
    % Moved to Biogenic Isoprene Chapter
    
  \subsection{Calculation of smearing effect}
    TODO: Smearing scale length, $\hat{S}$ formula, and results of calculations in here.
    As shown in \cite{Palmer2003}, smearing sensitivity can be calculated through multiple runs of the same model with the only difference being the isoprene emissions.
    I have run GEOS-Chem with and without E$_{ISOP}$ multiplied uniformly by 0.5, and the grid boxes with the most affected $\Omega_{HCHO}$ are those affected most by smearing.
    The smearing parameter  ($\hat{S}$) is defined as follows:
    \begin{equation}
      \hat{S} = \frac{\Delta \Omega_{HCHO}}{\Delta E_{ISOP}}
    \end{equation}
    TODO: Plot shows smearing parameter over Australia.
  
  \subsection{Calculations of uncertainty}
    There are several factors which need to be considered when looking at the uncertainty in emissions estimates.
    Things with their own inherent uncertainty include the modelled apriori, modelled relationship between HCHO and isoprene, and satellite measurements. 
    Important factors which need to be analysed for confidence in results include the steady state assumptions, filtering techniques for fire and human influences, and the regression model for determining the isoprene to HCHO yield.

    Uncertainty in satellite measurements is generally provided along with the data, although uncertainty introduced through AMF calculation needs to be determined to give a representation of the confidence in vertical column amounts.
    The measurement uncertainty is shown in section \ref{ch_HCHO:sec:OMI_uncertainty_calculation}, and amounts to $\sim X\%$. (TODO this number when calculated)
    
    Model uncertainty is difficult to accurately assertain, generally an analysis of the model compared to in-situ measurements is performed, however there are few of these measurements over Australia.
    TODO: find out how this is estimated in other papers, or else point to HCHO uncertainty and used some function of that.
    
    The uncertainty for HCHO to isoprene mechanisms TODO: how to do this?
  
  \subsection{Extrapolating the circadian cycle}
    Isoprene emissions occur with regular daily cycles caused by things like local temperature, sunlight, drought, and other environmental factors (TODO: find/cite eucalypt isoprene paper, daily cycle plot if can find).
    
    (TODO: following stuff, add some basic plots and error analysis eventually also)
    Using a model of the daily isoprene emissions fit to the offset determined by satellite HCHO based estimates, we produce a high temporal resolution isoprene emissions inventory.
    During days with more than one HCHO column measurement we can more confidently fit the cycle. 
    For example EOS AURA's OMI measurements from 2004 can be combined with MetOp-A's GOME2 after October 2006, with daily overpasses by OMI and GOME2 at 1345 and 0930 respectively.
    This allows a better retrieval of the daily amplitude of isoprene emissions.
    
  \subsection{Comparison with MEGAN}
    TODO: Direct comparison here, maps of differences for some metrics(monthly average,?). comparison of model run results using different inventory shown in section (reference here)

\section{New estimates affects on the Australian atmosphere}

