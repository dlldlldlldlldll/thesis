% CHAPTER 3 (probably)
% Isoprene Emissions

\chapter{Isoprene Emissions in Australia} % Main chapter title
\label{ch_isop} %better reference name like ch_isop

\section{GEOS-Chem isoprene mechanisms}
\label{ch_isop:sec:GEOSChemMechanisms}
  \subsetion{Outline}
    The isoprene reactions simulated by GEOS-Chem were originally based on (TODO: read+citet: Horowitz1998) (TODO: check wiki that this is true also).
    This involved blahblah(read that paper).
    The mechanism was subsequently updated by \citet{Mao2013}, who change the isoprene nitrates yields and add products based on current understanding as laid out in (TODO: read abstract of these and cite: Paulot2009a,Paulot2009b).
    The full current mechanism is described online at \url{http://wiki.seas.harvard.edu/geos-chem/index.php/New_isoprene_scheme} (TODO: check this).
    

%----------------------------------------------------------------------------------------
%	SECTION
%----------------------------------------------------------------------------------------
\section{Isoprene emissions estimation}
\label{ch_isop:sec:IsopreneEmissions}
  \subsection{Outline}
    Once we have vertical columns of biogenic HCHO ($\Omega_{HCHO}$) we can infer the local (grid space) isoprene emissions (E$_{ISOP}$) using effective formaldehyde yield from isoprene (S) (todo: cites with examples of this yield, link to yield table).
    This is simply expressed with the equation:
    \begin{equation} \label{ch_isop:eqn:isop_yield}
      \Omega_{HCHO} = S \times E_{ISOP} + B
    \end{equation}
    Where \textit{B} is the background HCHO.
    This works if there is fast HCHO yield, so that the effect of chemical transport is minimal.
    The background HCHO is assumed to be equal to HCHO measured in the remote pacific at the same time, however transport needs to be carefully handled.
    This yield can be calculated using CTMs such as GEOS-Chem, which was used by \citet{Millet2006} and produced a molar HCHO yield of 2.3.
    
    Isoprene quickly forms HCHO in the atmosphere when in the presence of high levels of NO$_X$.
    However, over Australia NO$_X$ levels are generally not high enough and we must take extra care that we can account for the transport or 'smearing' caused by slower HCHO formation.
    Smearing sensitive grid boxes within the model can be detected by running the model with two uniformly differing isoprene emission levels, then finding the grid boxes where the changed HCHO column is greater than can be attributed to local emission difference.
    Using equation \ref{ch_isop:eqn:isop_yield} at for two different isoprene emission levels:
    \begin{equation*}
      \hat{S} = \frac{\Delta~\Omega_{HCHO}}{\Delta~E_{ISOP}}
    \end{equation*}
    Consider halving the isoprene emitted globally and rerunning the model, if the local grid HCHO is reduced by much more than half (factoring yield) then you can infer sensitivity to non-local isoprene emissions.
    This sensitivity may change with local or regional weather patterns, as greater wind speeds will reduce the time any emitted compound stays within the local grid box.
    As such smearing sensitivity is both spatially and temporally diverse, shown in figure TODO: is a picture of the smearing sensitivity at blahpow.
   
    Once the smearing sensitive grid squares are filtered out, application of equation \ref{ch_isop:eqn:isop_yield} gives us an estimate of isoprene emissions across the nation.
    
    Most recently a \citet{Bauwens2016} undertook a similar process to what I am doing, although with slightly different focus, using the IMAGESv2 global CTM instead of GEOS-Chem.
    They calculate emissions which create the closest match between model and satellite vertical columns, and compare these postiori data with the apriori (satellite data) and independent data sets.
    (TODO: simple outline of what they did and how my focus is different, this paper will also need to be summarised in the LitReview)

    \subsection{Extrapolating the circadian cycle}
    Isoprene emissions occur with regular daily cycles caused by things like local temperature, sunlight, drought, and other environmental factors (TODO: find/cite eucalypt isoprene paper, daily cycle plot if can find).
    
    (TODO: following stuff, add some basic plots and error analysis eventually also)
    Using a model of the daily isoprene emissions fit to the offset determined by satellite HCHO based estimates, we produce a high temporal resolution isoprene emissions inventory.
    During days with more than one HCHO column measurement we can more confidently fit the cycle. 
    For example EOS AURA's OMI measurements from 2004 can be combined with MetOp-A's GOME2 after October 2006, with daily overpasses by OMI and GOME2 at 1345 and 0930 respectively.
    This allows a better retrieval of the daily amplitude of isoprene emissions.
    
  \subsection{Comparison with MEGAN}
    TODO: Direct comparison here, maps of differences for some metrics(monthly average,?). comparison of model run results using different inventory shown in section (reference here)

\section{Model comparison with and without satellite HCHO based inventory}

