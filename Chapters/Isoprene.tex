% CHAPTER 3 (probably)
% Isoprene Emissions

\chapter{TODO: move to biogenic isop chapter: Isoprene Emissions in Australia} % Main chapter title
\label{ch_isop}

\section{GEOS-Chem isoprene mechanisms(moved to Modelling chapter)}
  \label{ch_isop:sec:GEOSChemMechanisms}
  \subsection{Outline(moved)}
  \subsection{Emissions from MEGAN(moved)}

%----------------------------------------------------------------------------------------
%	SECTION
%----------------------------------------------------------------------------------------
\section{Isoprene emissions estimation}
\label{ch_isop:sec:IsopreneEmissions}

  \subsection{Outline}
    % What I do moved to biogenic isoprene outline
    
    % background comparison note sent to ???
    
    

  \subsection{HCHO Products and yield}
    % Yields and Tables moved to BiogenicIsoprene chapter
    
  \subsection{CAABA/MECCA yield}
    % Moved to Biogenic Isoprene Chapter
  
  \subsection{Calculation of Emissions}
    \label{ch_isop:sec:EmissionCalculation}
    % Moved to Biogenic Isoprene Chapter
    
  \subsection{Calculation of smearing effect}
    % moved to 
  \subsection{Calculations of uncertainty}
    

    
    
    Model uncertainty is difficult to accurately assertain, generally an analysis of the model compared to in-situ measurements is performed, however there are few of these measurements over Australia.
    TODO: find out how this is estimated in other papers, or else point to HCHO uncertainty and used some function of that.
    
    The uncertainty for HCHO to isoprene mechanisms TODO: how to do this?
  
  \subsection{Extrapolating the circadian cycle}
    Isoprene emissions occur with regular daily cycles caused by things like local temperature, sunlight, drought, and other environmental factors (TODO: find/cite eucalypt isoprene paper, daily cycle plot if can find).
    
    (TODO: following stuff, add some basic plots and error analysis eventually also)
    Using a model of the daily isoprene emissions fit to the offset determined by satellite HCHO based estimates, we produce a high temporal resolution isoprene emissions inventory.
    During days with more than one HCHO column measurement we can more confidently fit the cycle. 
    For example EOS AURA's OMI measurements from 2004 can be combined with MetOp-A's GOME2 after October 2006, with daily overpasses by OMI and GOME2 at 1345 and 0930 respectively.
    This allows a better retrieval of the daily amplitude of isoprene emissions.
    
  \subsection{Comparison with MEGAN}
    TODO: Direct comparison here, maps of differences for some metrics(monthly average,?). comparison of model run results using different inventory shown in section (reference here)

\section{New estimates affects on the Australian atmosphere}

