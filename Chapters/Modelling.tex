% CHAPTER 2 (probably)
% MODELLING

\chapter{Modelling with GEOS-Chem} % Main chapter title
\label{Model} %better reference name?

%----------------------------------------------------------------------------------------
%	SECTION
%----------------------------------------------------------------------------------------
\section{GEOS-Chem framework}

\section{GEOS-Chem isoprene mechanisms}
  \label{ch_isop:sec:GEOSChemMechanisms}
  \subsection{Outline}
    The isoprene reactions simulated by GEOS-Chem were originally based on \cite{Horowitz1998}.
    This involved simulating NO$_X$, O$_3$, and NMHC chemistry in the troposphere at continental scale in three dimensions, with detailed NMHC chemistry with isoprene reactions and products.
    The mechanism was subsequently updated by \citet{Mao2013}, who change the isoprene nitrates yields and add products based on current understanding as laid out in \citet{Paulot2009a,Paulot2009b}.
    Further mechanistic properties, like isomerisation rates, are based on results from four publications: cite{Crounse2011,Crounse2012,Peeters2010,Peeters2011}.
    (TODO: check abstracts Peeters papers).
    \cite{Crounse2011} examines the isomerisations associated with the oxidation of isoprene to six different isomers (ISO$_2$) formed in the presence of oxygen: isoprene $ + OH \rightarrow^{O_2} $ ISO$_2$.
    They determine rates and uncertainties involved in these reactions, and study the rate of formation of C$_5$-hydroperoxyaldehydes (HPALDs) by isomerisation.
    \cite{Crounse2012} examine the fate of methacrolein (MACR), one of the products of isoprene oxidation. 
    Prior to this work MACR oxidation chamber studies were performed in high NO or HO$_2$ concentrations, giving peroxy lifetimes of less than 0.1~s.
    In most environments this is not the case, and MACR products over various NO concentrations and peroxy radical lifetimes are determined in their work.
    \cite{Peeters2010} examine photolysis of hydroperoxy-methyl-butenals (HPALDs, produced by isoprene isomerisation), which regenerates OH levels in areas with high isoprene emissions.
    Additionally, photolysis of photolabile peroxy-acid-aldehydes creates OH and improved model aggreement with continental observations.
    The OH and HPALD interactions are central to maintaining the OH levels in pristine and moderately polluted environments, which makes isoprene both a source and a sink of OH TODO: cite and DL;\url{http://www.nature.com/ngeo/journal/v5/n3/full/ngeo1405.html}.
    
    Formation of isoprene nitrates have an effect on ozone levels through NO$_X$ sequestration, and the yields and destinies of these nitrates is analysed in \citet{Paulot2009a}. 
    They use anion chemical ionization mass spectrometry (CIMS) to determine products of isoprene photooxidation.
    In a chamber with clean air and high NO concentrations, isoprene photooxidation is initially driven by OH addition, followed by NO$_X$ chemistry (150~min - 600~min), and finally HO$_X$ dominated chemistry.
    The yields of various positional isomers of isoprene nitrates is estimated, and pathways of their oxidation products is shown and used in the GEOS-Chem isoprene mechanism \citep{Paulot2009a,Mao2013}. 
    
    In low NO$_X$ conditions, isoprene oxidises to yield 70\% hydroxyhydroperoxides (ISOPOOH), which then oxidises to create dihydroxyperoxides (IEPOX) with OH recycling maintaining the OH levels in the atmosphere \citep{Paulot2009b}.
    In older models isoprene produced ISOPOOH which then titrated OH, however, the loss of OH has not been seen in measurements \citep{Paulot2009b,Mao2013}.
    The isoprene mechanism in GEOS-Chem has been updated to include OH regeneration from oxidation of epoxydiols and slow isomerisation of ISOPO$_2$ \citep{Mao2013}.
    
    Under high NO$_X$ conditions, isoprene undergoes OH addition at the 1 and 4 positions, becoming $\beta$ (71\%) or $\delta$ (29\%) hydroxyl peroxy radicals (ISOPO$_2$). 
    The $\beta$-hydroxyl reacts with NO$_X$ and produces HCHO (66\%), methylvinylketone (40\%) (MVK), methacrolein (26\%), and $\beta$-hydroxyl nitrates (6.7\%) (ISOPNB).
    The $\delta$-hydroxyl reacts with NO to form $\delta$-hydroxyl nitrates (24\%) (ISOPND), and ISOPNB (6.7\%).
    ISOPNB and ISOPND yield first generation isoprene at 4.7\% and 7\% respectively.
    
    Under low NO$_X$ conditions, ISOPO$_2$ may react with HO$_2$ to form ISOPOOH.
    In this case there is also production of HCHO (4.7\%), MVK(7.3\%), and MACR (12\%).
    As stated in earlier; most ISOPOOH will form IEPOX (epoxydiols) after reacting with OH and lead to OH regeneration.
    The other mechanism in low NO$_X$ environments is unimolecular isomerisation of ISOPO$_2$.
    This leads to production of hydroperoxyaldehydes (HPALDS), which generally photolyse and have an OH yield of 100\%.
    \citet{Mao2013} show that a lower (factor of 50) rate constant for ISOPO$_2$ isomerisation leads to better organic nitrate aggreements with ICARTT. 
    
    This update leads to more accurate modelling of OH concentrations, especially in low NO$_X$ conditions common in remote forests.
    Prior to \citet{Mao2012}, measurements of OH in high VOC regions may have been up to double the real atmospheric OH levels, due to formation of OH inside the instrument.
    \citet{Mao2012} examine an upgraded method of measurement, and compare these against a regional atmospheric chemistry model (RACM2), with the OH recycling updates from \citet{Paulot2009b} as discussed in prior paragraphs.
    
    The updates to isoprene chemistry by \citet{Mao2013}, and those shown in \cite{Crounse2011,Crounse2012} are the last before version 11, which was not used in this work.
    
    The full current mechanism is described online at \url{http://wiki.seas.harvard.edu/geos-chem/index.php/New_isoprene_scheme}.

  \subsection{Emissions from MEGAN}
    MEGAN simulates biogenic emissions of various gases including isoprene, based on various meteorological, land cover, and plant type parameterisations.
    
    One of the important parameters in Australia is the soil moisture activity factor($\gamma_{SM}$), which can have large regional affects on the isoprene emissions \citep{Sindelarova2014,Bauwens2016}.
    Generally if soil moisture is too low, isoprene emissions stop \citep{Pegoraro2004,Niinemets2010}, however in many Australian regions the plants may be more adapted to lower moisture levels. (TODO: Find cites for this - talk from K Emerson at Stanley indicated this)
    GEOS-Chem runs MEGANv2.1, which has three possible states for isoprene emissions based on the soil moisture ($\theta$):
    \begin{align*}
      \gamma_\mathrm{SM} & = 1 && \theta > \theta_1 \\
      \gamma_\mathrm{SM} & = (\theta-\theta_w)/\Delta\theta_1  && \theta_w < \theta < \theta_1 \\
      \gamma_\mathrm{SM} & = 0 && \theta < \theta_w \\
    \end{align*}
    where $\theta_w$ is the wilting point, and $\theta_1$ determines when plants are near the wilting point.
    The wilting point is set by a land based database from \citet{Chen2001}, while $\theta_1$ is set globally based on \citet{Pegoraro2004}.
    
    In GEOS-Chem the emissionscan be globally multiplied by a constant factor, which was performed to determine the smearing and sensitivity.
    By running the model two extra times, with the biogenic emissions set to zero and one half, while other parameters remain unchanged, the general affects of isoprene emissions which the model undergoes can be determined.