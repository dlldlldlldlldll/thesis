
% Background Chapter
%%%
% New organisation:
%
% Introduction and Literature Review
%   Tropospheric ozone and air quality . . . . .
%   Isoprene and other VOCs . . . . . . . . . . .
%   Formaldehyde (HCHO) . . . . . . . . . . . .
%   Models . . . . . . . . . . . . . . . . . . . . .
%   Aims?
%%%%%%
\chapter{Introduction and Literature Review} % Chapter Title
\label{LR}
  
\section{Tropospheric ozone and air quality}
  \label{LR:O3andAQ}
  \subsection{Air Quality}
    \label{LR:O3andAQ:AQ}
    %%% AUSTRALIA
    
    % Moved to intro    
    
    \subsubsection{Ozone}
      %%% OZONE
      
      
      
    
    \subsubsection{Particulate matter and SOA}
      
    
    
    \subsubsection{Factors influencing ozone and PM}
      %\label{LR:O3andAQ:AQfactors}
      
    \subsubsection{How do we measure air quality?}
      
      
      
      
      
  \subsection{Ozone transported from the stratosphere}
    
    
    
    
    
    
    
    
  

  \subsection{Ozone formed in the troposphere}
    \label{LR:O3andAQ:BiogenicOzonePrecursors}
    
    
    
\section{Hydroxyl (OH) and other radicals}
  \label{LR:Radicals}
  
  
  
  
  

\section{Isoprene and other VOCs}
  \label{LR:VOCs}
  \subsection{What are VOCs}
    

    
  \subsection{What do they Do?}
    
    
    
  \subsection{Isoprene Cascade}
  
    
  \subsection{How and where do we measure them?}
    

  
  
  \subsection{Emissions estimates}
    \label{LR:VOCs:EmissionsEstimates}
    
    
  
\section{Formaldehyde (HCHO)}
  \label{LR:HCHO}
  
  %Paragraphs moved to Intro
  \subsection{How HCHO is measured}
  
  \subsection{Satellite Inversion}
  
  \subsection{Satellite HCHO detection}
    \label{LR:HCHO:SatelliteDetection}
    TODO: Refactor this section so it's readable
     
    
      
    
    
    \subsubsection{Satellite uncertainties}
      
    
    
  
    
\section{Models}
  \label{LR:Models}
  \subsection{How can models help}
    
    
    
    
    
  \subsection{Relevant model frameworks}
  \label{LR:Models:frames}
    
    % Outline of ACM
    Atmospheric chemistry models (ACMs) require various inputs and can be sensitive to ozone and oxidative parameterisations. 
    TODO: read more Christian 2017,
    TODO: put some more generic ACM info here.
    
    \subsubsection{Box models} 
    \label{LR:Models:frames:box}
    
      
      
    \subsubsection{Chemical transport} %% eg. GEOS-Chem
      
      
      The Model of Emissions of Gases and Aerosols in Nature (MEGAN) is one of the more commonly used natural emission models \citep{Monks2015}. (TODO: more cites which say this/use MEGAN)
    
    \subsubsection{Land based emissions} %% EG MEGAN
      % TODO: Overview of land based emissions modelling?
      
      
      

    \subsubsection{Radiative transfer} %% EG LIDORT
      %TODO: Lidort example?
      TODO: Lidort example?
    
  \subsection{Factors affecting isoprene emissions estimates}

      \cite{Marais2014} examine factors affecting isoprene emissions, showing how emissions are sensitive to various environmental factors.
      Their work used MEGAN \citep{Guenther1995} and GEOS-Chem to look at how these factors affect surface ozone and particulate matter in Africa.
      One of the important uncertainties seen in MEGAN within this work is the isoprene emissions due to plant type.
      Canopy level isoprene measurements are made using relaxed eddy accumulation (REA) at several sites in Africa.
      One plant type near a measurement site emits more than other species and it's actual distribution on a larger scale is completely unknown - leading to possible overestimations in MEGAN.
      Current emissions estimates require more validation against observations, and recently a comparison of two major VOC models (MEGAN and ORCHIDEE) was undertaken by \cite{Messina2016} reiterating this requirement.
      In their work they examine model sensitivities and show that the important parameters are leaf area index (LAI), emission factors (EF), plant functional type (PFT), and light density fraction (LDF).
      There is high uncertainty in LAI and EF, which require more or improved measurements at the global scale.
      LDF paramterisation needs improvement and these models require more PFTs.
      Global emissions inventories like MEGAN suffer from large extrapolations which introduce uncertainties \citep{Miller2014}.

      \cite{Emmerson2016} analyse EF sensitivity of a high resolution model of atmospheric chemistry over southeast Australia, comparing isoprene and monoterpene emissions against 4 separate campaigns.
      They show that the effect on total emissions is roughly linear and that no blanket EF changes are appropriate for all regions/seasons.
      They also mention that Australian eucalypt emissions are based on samples from young trees, which may emit more isoprene than older trees.
      
      \cite{Stavrakou2014} examined modelled Asian emissions and altered model parameters for temperature, plant type emission factors, incoming solar radiation (insolation) intensity, land use changes, and palm tree forest expansion.
      Changes were constrained by a network of radiation measurements and some experiments with south east Asian forest emissions - and led to reduction in isoprene emissions by a factor of two over the region.
      The Asian region is also shown to have a strong correlation with the Oceanic Niño Index (ONI), with positive anomalies associated with El Niño.
      In the last 20 years anthropogenic emissions of VOCs have been increasing while biogenic VOC emissions have decreased due to rapid economic growth and lower annual temperatures \citep{Stavrakou2014, Kwon2017}.
      
      %Temperature has a strong exponential relationship with isoprene emissions, and can be readily seen in comparisons to a major isoprene product HCHO. 
  
  % TODO: Reading up to here
  \subsection{Uncertainties}
    \label{LR:Models:Unc}
    Here I will attempt to list and partially explain the major uncertainties models have in relation to  VOCs, SOAs, and ozone. 
    TODO: Is this a good idea or should I put any pertinent uncertainties with the associated work/descriptions?
    
    \subsubsection{Emissions Inventories}
      % Emissions Inventories 
      Using different emissions inventories in an ACM can have large impacts on the simulation.
      Natural (biogenic or pyrogenic) and human driven (anthropogenic) emissions often drive a large fraction of atmospheric oxidation and radical chemistry, especially in the continental boundary layer.
      \cite{Zeng2015} examine the affects on CO and HCHO when running simulations with two different inventories.
      TODO: find where I took notes about Zeng2015 and put them here.
    
    %% GEOS-Chem resolution uncertainties
    \subsubsection{Resolution}
      \label{LR:Models:Unc:Resolution}
      GEOS-Chem simulations are somewhat sensitive to the resolution at which you run.
      For example: \cite{Wild2006} show that reduced resolution increases OH concentrations and ozone production rates.
      \cite{Christian2017} find small changes in OH ($<10$\%) in OH, HO$_2$ and ozone concentrations local to the north american arctic, when changing from 4 by 5 to 2 by 2.5\degr resolution, however they continue at lower resolution to save computational time.
      
      For many global scale analyses, errors from resolution are less important than those from chemistry, meteorology, and emissions (\cite{Christian2017}).
      
    
    % Transport uncertainties?
    \subsubsection{Transport}
      \label{LR:Models:Unc:Transport}
      TODO: Literature showing transport uncertainties or lack thereof     
      %TODO: examples of transport uncertainties
    
    \subsubsection{Chemistry mechanisms}
      \label{LR:Models:Unc:Chemistry}
      %% GEOS-Chem Ozone uncertainties 
      There is still much work to be done in models to correctly simulate the various precursors to HCHO.
      Often HCHO is used as a way of checking if these precursors are correctly modelled since HCHO measurements are more readily available (for instance from satellites).
      GEOS-Chem has recently been analysed for sensitivity for ozone along with oxidants (OH and HO$_2$) \citep{Christian2017}.
      \cite{Christian2017} found that GEOS-Chem ozone was most sensitive to NO$_2$ photolysis, the $NO_2 + OH$ reaction rate, and various emissions.
      They used GEOS-Chem v9-02, with $4^{\circ} \times 5^{\circ}$ resolution, and while the low resolution adds errors in OH concentrations and O$_3$ production rates, the errors from chemistry, meteorology, and emissions are much larger.

      \cite{Marvin2017} suggest that isoprene mechanisms in several contemporary models (including GEOS-Chem) are inadequate. 
      They show that for a specific measurement campaign, the HCHO concentrations are underestimated in a way that can not be easily fixed through rate constant changes.
      Recently \cite{Marvin2017} compared five global ACMs isoprene mechanisms by evaluating simulated HCHO mixing ratios compared to in situ measurements from the Southeast Nexus (SENEX) aircraft campaign (in southeastern USA).
      They compared five models (GEOS-Chem, CB05, CB6r2, MCMv3.2, and MCMv3.3.1) and found all of them underestimated the HCHO concentrations (by $15 - 30\%$).
    
    \subsubsection{Clouds}
      \label{LR:Models:Unc:Clouds}
      One of the major uncertainties in chemical, climate, radiation, and weather models is cloud formation and dynamics.
      Clouds are remarkably complex at a much finer scale than can be accurately modelled by global chemistry models (with current processing power).
      Globally over half (50-60\%) of the world is covered by clouds, with $\sim10\%$ of them being rain-clouds \citep{Kanakidou2005}.
      Wet scavenging performed in clouds not only depends on large scale cloud processes, but also on the microphysics of aerosols being scavenged, differing between aerosol sizes and hygroscopic properties.
      
    \subsubsection{Soil Moisture}
      \label{LR:Models:Unc:SoilMoisture}
      Australia has a unique climate, along with soil moisture, clay content and other important properties which affect VOC emissions.
      These properties are poorly understood in Australia due to the continents size and the relative sparsity of population centres, which make many areas very difficult or expensive to reach.
      Soil moisture plays an important role in VOC emissions, as trees under stress may stop emitting various chemicals. 
      This is especially true for Australia due to frequent droughts and wildfires.
      The argument for improved understanding of land surface properties, specifically soil moisture, is an old one\citep{Mintz1982, Rowntree1983, Chen2001}. 
      \cite{Rowntree1983} show how quickly soil moisture anomalies affect rainfall and other weather systems, while \cite{Chen2001} specifically show how important fine scale soil moisture information is when modelling land surface heat flux, and energy balances.
      Modelled emissions are sensitive to soil moisture, especially near the soil moisture threshold (or wilting point), below which trees stop emitting isoprene and other VOCs completely as they can no longer draw water \citep{Bauwens2016}.
      MEGAN accounts for soil moisture by applying it as an emission factor (EF) which scales the emission rate of various species.
      \cite{Sindelarova2014} show reductions in modelled Australian isoprene emissions of 50\% when incorporating soil moisture in MEGAN estimates. 
      
      Droughts affects can be difficult to measure, as it is a multi-scale problem which affects various aspects of the land-air interface including plant emissions and dry deposition (\cite{Wang2017}).
      The Standardised Precipitation Evapotranspiration Index (SPEI) is a measure of drought using TODO \cite{SPEI_website}.
      This product covers 1901 - 2011, and uses the average over that period as the background, in order to compare drought stressed regions against those with sufficient or excess water \cite{SPEI_website}.
      
