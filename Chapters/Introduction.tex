\chapter{Introduction and Literature Review} % Chapter Title
\label{LR}

\section{The atmosphere}
\label{LR:Atmos}
  % Overarching description of atmosphere
  The atmosphere is made up of various gases held to the earths surface by gravity. 
  These gases undergo transport on all scales, from barbeque smoke being blown into your face to smoke plumes from forest fires travelling accross the world and depositing in the antarctic snow.
  They take part in various chemical reactions along the way, largely driven by solar input and interactions with eachother.
  Various chemicals are lofted into the atmosphere by soil, trees, factories, cars, seas and oceans, you name it.
  They are also deposited back to the surface both directly and in rain drops.
  
  % Air
  Mostly the atmosphere is made up of nitrogen (N$_2$: $\sim 78\%$), oxygen (O$_2$: $\sim 21\%$), and argon (Ar: $\sim 1\%$).
  Water (H$_2$O) ranges from $0.001$ to $1\%$ depending on evaporation and precipitation.
  Beyond these major constituents the atmosphere has a vast number of \textit{trace gases}, including carbon dioxide (CO$_2$: $\sim 0.4\%$), Ozone (O$_3$: $.000001$ to $0.001\%$), and methane (CH$_4$: $\sim 0.4\%$) \cite[][Ch. 2]{BrasseurJacob2017}.
  Trace gases in the atmosphere can have a large impact on living conditions.
  They react in complex ways with other elements (anthropogenic and natural), affecting various ecosystems upon which life depends.
  
  % TODO: Pressure
  Most of the atmosphere ($\sim 85\%$) is within 10~km of the earths surface.
  This is due to air pressure, which decreases logarithmically with altitude.
  

  % TODO: Structure lead in
  
  \subsection{Structure}
  \label{LR:Atmos:Struct}
    
    % TODO: Boundary layer
    
    % TODO: Troposphere
    
    % TODO: Stratosphere
    
  % Chemistry
  \subsection{Chemistry}
  \label{LR:Atmos:Chem}
    % Oxidation and Radicals
    
    % Photolysis stuff
    Ozone is also a very important substance for formation of radicals (NO$_3$, OH) in the troposphere through photolysis in the presence of water.
    
    
    
    % Ozone lead in
    Ozone in the lower atmosphere is a serious hazard that causes health problems \citep{Hsieh2013}, damages agricultural crops worth billions of dollars \citep{Avnery2011,Yue2017}, and increases the rate of climate warming \citep{IPCC_2013_chap8}.

\section{Ozone}
\label{LR:O3}
  %TODO What is ozone
  Ozone (O$_3$) is mostly located in the stratosphere, where it helpfully prevents much of the shorter wave length solar radiation from reaching the earth's surface (ie UV light).
  However around 11\% of the total column of ozone is located in the troposphere (TODO: cite), where it has several deleterious effects.
  Around 5 to 20 percent of all air pollution related deaths are due to ozone (\cite{Monks2015}).
  In the short term, ozone concentrations of $\sim$50-60~ppbv over eight hours or $\sim$80~ppbv over one hour are agreed to constitude a human health hazard \citep{Ayers2006,Lelieveld2009}. 
  Long term exposure to lower levels cause problems with crop loss and ecosystem damage \citep{Emberson2003}, and both short and long term concentrations may get worse in the future \citep{Lelieveld2009, Stevenson2013}.
  Further tropospheric ozone enhancements are projected to drive reductions in global crop yields equivalent to losses of up to \$USD$_{2000}$ 35 billion per year by 2030 \citep{Avnery2011}, along with detrimental health outcomes equivalent to $\sim$\$USD$_{2000}$11.8 billion per year by 2050 \citep{Selin2009}.
  Recently \cite{Yue2017} showed that the net effect of near-surface ozone on is a $\sim 14\%$ decrease in net primary productivity (NPP) in China, which could be reduced by $\sim 70\%$ with drastic measures by 2030.
  
  %TODO Structure of ozone layer chapman eqn etc.
  
  
  % TODO things that affect ozone concentrations 
  Smoke plumes from biomass burning can carry ozone precursors, creating higher ozone concentrations downwind of the plume's source.
  Fire emissions include a range of chemicals and each year the affects of fire or burning seasons blanket the northern and southern hemispheres independently.
  
  
  \subsection{Stratosphere to troposphere transport}

  
  \subsection{Chemical production}
    The tropospheric ozone concentrations rely on climate and ozone precursor emissions; including NO, NO$_2$, CO, VOCs, and HCHO \citep{Atkinson2000, Young2013, Marvin2017}. 
    Ozone predictions are uncertain and difficult due to the vagaries of changing climate which affects both transport, deposition, destruction, and plant based precursor emissions.
    All of these processes are tightly coupled and difficult to accurately model, as they depend on uncertain assumptions such as CO$_2$ dependency \citep{Young2013}.
    Even with all the work done in the prior decades there remains large uncertainties about ozone precursors in the troposphere \citep{Mazzuca2016}.
    
    NO$_X$ (NO, NO$_2$) has a complex non-linear relationship with both ozone and VOCs.
    The role of NO$_X$ on VOC oxidation is discussed in more detail in Section \ref{LR:VOCs:IsopCascade}.
    

\section{VOCs}
\label{LR:VOCs}
  % What they are
  %%% Important factors affecting PM and Ozone
  Volatile Organic Compounds (VOCs) are an important driver of atmospheric processes, especially near forests.
  The major source of VOCs in the atmosphere is biogenic, with around 90\% of emissions (globally) coming from natural sources \citep{Guenther1995,Guenther2006, Millet2006}.
  Global non-methane VOC (NMVOC) levels are estimated at 85~\%, 13~\%, and 3~\% from biogenic, anthropogenic, and pyrogenic sources respectively \citep{Kefauver2014}.
  Methane and isoprene each comprise around a third of the global total emissions of VOCs (\cite{Guenther2006}).
  However, methane is relatively long lived (years) and is well mixed in the atmosphere while isoprene levels are very volatile and spatially diverse due to a life time of around an hour.
  This means that methane measurements and concentrations are relatively well understood, while NMVOCs are less so.
  
  % Impacts (AQ, Oxidation)
  %%% PM and SOA
  PM in the atmosphere is a major problem, causing an estimated 2-3 million deaths annually \citep{Hoek2013, Krewski2009, Silva2013, Lelieveld2015}. 
  Aerosols are suspended particulates and liquid compounds in the atmosphere, of which particulate matter (PM) is an important subset.
  Fine particulate matter (PM$_{2.5}$) penetrates deep into the lungs and is detrimental to human health.
  Some PM comes from small organic aerosols (OA) emitted in the particulate phase and referred to as primary OA (POA).
  A substantial amount of PM is due to gaseous organic compounds transforming in the troposphere leading to what's known as secondary OA (SOA) \citep{Kroll2008}.
  Formation of SOA is generally due to VOC oxidation and subsequent reactions, while removal from the atmosphere is largely due to wet or dry deposition, and cloud scavenging \citep{Kanakidou2005}.
  
  % Isoprene subsection lead in
  Of these VOCs, isoprene has major impacts and is relatively uncertain.
  
  \subsection{Isoprene}
  \label{LR:VOCs:Isop}
    
    
  \subsection{Emissions}
  \label{LR:VOCs:Emissions}
    
  
    % Lead in for HCHO section
    One of the major products of isoprene chemistry is HCHO.
    \subsection{Isoprene Cascade}
    \label{LR:VOCs:IsopCascade}
      
\section{HCHO}
\label{LR:HCHO}
  % What is HCHO:
  HCHO, aka methanal, methyl aldehyde, or methylene oxide, is of the aldehyde family.
  HCHO is an OVOC which is toxic, allergenic, and a potential carcinogen. 
  It is dangerous at low levels, with WHO guidelines for prolonged exposure at 80~ppb.
  HCHO is used as an adhesive in plywood, carpeting, and in the creation of paints and wallpapers.
  Emissions in enclosed spaces can build up to dangerous levels, especially if new furnishings are installed (\cite{Davenport2015}).
  At global scales HCHO in furniture is less important, as concentrations are driven by photochemical reactions with methane and other VOCs.
  
  %Sources
  
  
  %TODO How measured (in-situ, satellite)
  
  \subsection{Satellite measurements}
  \label{LR:HCHO:Sat}
  
\section{Modelling}
\label{LR:Models}

\section{Australia}
\label{LR:Aus}
  % Description of uniqueness
  Australia is largely covered by environments which are not heavily influenced by human activity.
  These regions are natural (biogenic) sources of the trace gases (those which make up less than 1\% of the earth's atmosphere).
  Biogenic emissions affect surface pollution levels, potentially enhancing particulate matter (PM) and ozone levels.
  Due to the lack of in-situ ground based measurements, estimates of VOC emissions are uncertain, with large scale extrapolation required \cite{Millet2006}.
  
  % voc estimates
  VOC emission estimates are based on (and highly sensitive to) many factors, including plant type and soil moisture \citep{Guenther1995}, neither of which are well characterised in Australia \citep{Sindelarova2014, Bauwens2016}.
  Changes in parameterisation of soil moisture in the Model of Emissions of Gases and Aerosols from Nature (MEGAN, \cite{Guenther1995}) lead to massive changes in Australian isoprene emission estimates \citep{Sindelarova2014}.
  This has an compounding effect on the large uncertainties of biogenic VOC emissions \citep{Guenther2000, Millet2006}.
  
  Emissions of VOCs are not well understood, especially in Australia, where they are largely extrapolated.  
  Since many Australian cities are on the edge of regions with rich VOC emissions, it is very important to clarify the quantity, type, and cause of VOC emissions.
  The existence of satellite data covering remote areas provides an opportunity to improve VOC emissions estimates leading to more robust models of global climate and chemistry.
  Understanding of emissions from these areas is necessary to inform national policy on air pollution levels.
  
  %Transport stuff
  Biomass burning in southern Africa and South America has previously been shown to have a major influence on atmospheric composition in Australia \citep{Oltmans2001, Gloudemans2006, Edwards2006}, particularly from July to December \citep{Pak2003, Liu2016}.
  
\section{Aims}
\label{LR:Aims}
TODO: outline of aims here (FIND THESE THEY ARE SOMEWHERE)

\section{Data Access}
TODO: ADD MORE HERE
\label{LR:DataAccess}
\begin{description}
  \item[OMNO2d] Daily satellite NO$_2$ product downloaded from \url{https://search.earthdata.nasa.gov/search}, DOI:10.5067/Aura/OMI/DATA3007
  
  \item[SPEI] Monthly standardised precipitation evapotranspiration index (metric to determine drought stress) downloaded from \url{http://hdl.handle.net/10261/153475} with DOI:10.20350/digitalCSIC/8508
  
  \item[OMHCHO] Satellite swaths of HCHO slant columns downloaded from TODO, with DOI TODO
  
\end{description}