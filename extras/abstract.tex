
%----------------------------------------------------------------------------------------
%	ABSTRACT PAGE
%----------------------------------------------------------------------------------------
\begin{abstract}
  \addchaptertocentry{\abstractname} % Add the abstract to the table of contents
  %The Thesis Abstract is written here (and usually kept to just this page). The page is kept centered vertically so can expand into the blank space above the title too\ldots
  
  Atmospheric chemistry transport models (CMTs) have uncertain biogenic emissions in Australia, which detrimentally affects the confidence of outputs for many important chemical species. 
  Isoprene, formaldehyde, and ozone in the troposphere are linked by oxidative chemistry and are all important to air quality, climate, and radiation budgets.
  My thesis has three aims: recalculating formaldehyde amounts over Australia seen by satellite using recent a priori modelled profiles from a global CTM (GEOS-Chem), determination of isoprene emissions using modelled formaldehyde yields along with satellite formaldehyde amounts, and attribution of ozone in the troposphere (from either chemical production following isoprene emissions or transported ozone plumes from the stratosphere).
  Model and satellite datasets are combined for each aim in this thesis and where possible outputs are compared against campaigns which measured isoprene, formaldehyde, or ozone concentrations.
  Combining or comparing these disparate datasets requires careful analysis due to the structural, temporal, and spatial differences between the datasets.
  
  Formaldehyde seen by satellite over Australia is calculated based on modelled a priori vertical distributions using both CTMs and radiative transfer models.
  In order to compare satellite formaldehyde products against models or other measurements, corrections are required to remove the influence of the a priori profile.
  Formaldehyde from OMI, on board the NASA AURA satellite is recalculated using GEOS-Chem concentration profiles over Australia.
  Impacts on formaldehyde levels from anthropogenic and pyrogenic sources require identification and filtering in order to determine the biogenic footprint and allow an unbiased isoprene emissions quantification.
  
  Isoprene is largely predominantly emitted by trees and shrubs during the daytime.
  The subsequent oxidation reactions form formaldehyde which lasts long enough in the atmosphere to form an equilibrium and is measurable by satellite.
  Using a simple linear model and assuming low transport and a steady state allows an estimate of the yield of formaldehyde from isoprene emissions.
  This yield is modelled over Australia and then applied to the recalculated satellite formaldehyde in order to estimate isoprene emissions.
  This technique is called a top-down emissions estimate, and is used to improve isoprene emissions estimates without the need for extensive measurement campaigns.
  
  The majority of tropospheric ozone (tO3) is formed through chemical reactions involving nitrogen oxides, the hydroxy radical, and volatile organic compounds (such as isoprene).
  The second most abundant source of tO3 is the stratosphere, which occasionally mixes into the troposphere bringing ozone rich air masses down towards the earths surface.
  Using a Fourier filter on ozone profiles measured from ozonesondes, and analysing the local weather patterns and ozone seasonality, we derive an estimate of how much tO3 enhancement is occurring due to these stratospheric intrusions.
  
  
  
\end{abstract}
