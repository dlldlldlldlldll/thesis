
\documentclass[11pt]{article} % The class file specifying the document structure
\usepackage[utf8]{inputenc} % Required for inputting international characters
\usepackage[T1]{fontenc} % Output font encoding for international characters
\usepackage{palatino} % Use the Palatino font by default
\usepackage{graphicx} % graphicx package for images
\usepackage{url} % package for URL displaying
\usepackage[backend=bibtex,bibencoding=ascii,style=authoryear,sorting=none,natbib=true]{biblatex}
%\usepackage[backend=biber,bibencoding=ascii,style=authoryear,sorting=none,natbib=true]{biblatex} 

\bibliography{references_somesummaries}

\begin{document}

%----------------------------------------------------------------------------------------
%	TITLE PAGE
%----------------------------------------------------------------------------------------

\begin{titlepage}
\begin{center}
Summary paragraphs of papers + some quotes

\end{center}
\end{titlepage}

\tableofcontents % Prints the main table of contents
\listoffigures % Prints the list of figures
\listoftables % Prints the list of tables

%----------------------------------------------------------------------------------------
%	PAPERS AND THEIR SUMMARIES
%----------------------------------------------------------------------------------------

\section{Borchi 2007:}


\section{Liu 2015: Origins of tropospheric ozone interannual variation over Réunion: A model investigation}
  \citet{Liu2015}:
  Ozonesondes released at reunion island as part of SHADOZ(\url{http://croc.gsfc.nasa.gov/shadoz/}) are analysed against GMI-CTM grid box profiles matching sonde release dates.
  50\% increase in tropospheric ozone from 1992 - 2011 in JJA months based on ozonesonde measurements.
  ``Far fewer studies have focused on the long-term behaviour of ozone in the Southern Hemisphere owing to a lack of suitable observations.''
  Sparse sampling does not fully capture Interannual Variability (IAV) in the upper troposphere shown by the model.
  
  Stratospheric influence over Reunion of the tropospheric ozone is important, although model resolution and sonde sparcity make analysis more difficult.
  ``Increasing emission over southern Africa appear to affect only the lower troposphere'', with an increasing trend in the middle and upper troposphere during austral winter left unexplained.
  The widening tropical belt may also be increasing the stratospheric influence over Reunion.

\section{Mze 2010:}
  \citet{Mze2010}
  ``The tropical stratosphere is, however, a zone where it is difficult to measure ozone by satellite experiments, due to increased Rayleigh atmospheric attenuation, high altitude clouds, low temperature, high humidity and dense aerosols (Borchi et al., 2007)''.

\section{Thompson 2015: }
  
% bibliography
\printbibliography[heading=bibintoc]

\end{document}  


