\chapter{Outline}

  \section{Overview}
    In this thesis I will combine satellite and ground based atmospheric measurements with chemical transport modelling to clarify the impact of Australian natural emissions on atmospheric composition and chemistry.
    This broadly includes O$_3$, NO$_X$, VOCs, HO$_X$, and the various processes which influence their composition and transport.
    Satellite measurements of HCHO, calculations of VOC emissions, HCHO and VOC atmospheric distributions, variations, and chemistry will be laid out with isoprene calculations being the main theme.
    The techniques used to analyse and create the products I use in my thesis will be explained.
    A background of HCHO modelling and MEGAN and GEOS-Chem will be split between here and the introduction to chapter 2.

    A review of the literature on several topics is included with a view to touch on all the possible factors affecting my own work.
    Ozone and ozone precursors, along with the toxic effects and future projections of tropospheric ozone will be discussed.
    This leads into my \textbf{first chapter: ``Stratosphere to Troposphere Transport of ozone''}.
    Here I analyse a dataset of ozonesondes, and calculate how many STTs occur in the high southern latitudes.
    An overview of my work categorising Stratospheric transport over the southern high latitudes will be the focus of this chapter as well as a paper I hope to submit next month.
    
    An in depth review of how the GEOS-Chem model compares against ozonesondes will also be part of this chapter.
    GEOS-Chem is used to estimate how much tropospheric ozone is due to these STT events.
    These processes are some portion of the tropospheric ozone source - ties into another source: VOCs

    My \textbf{second chapter: ``HCHO total columns in Australia''} is focussed on calibrating a grided HCHO product from the OMI satellite measurements over Australia between January 2005 and April 2013.
    GOES-Chem and MEGAN model histories will form some part of the background as they have been used several times to undertake similar inversions of VOC emissions.
    The process follows that of several other publications, however I focus on Australian emissions (largely unmentioned elsewhere) and parameters including precursor yields and grid resolution.
    Estimation of uncertainties and sensitivities to various factors including model grid resolution, fire, and anthropogenic masking operations is performed.
    Finally validation is undertaken using in-situ measurements of HCHO from an FTIR instrument on the roof of the chemistry building at the University of Wollongong.
    
    My \textbf{third chapter ``Isoprene emissions in Australia''} uses the HCHO product developed in chapter 2 along with various modelled parameters to estimate the emissions of isoprene (and possibly other BVOCs) through a simple linear steady state model, along with an analysis of the assumptions required for this model.
    Ideally the isoprene emissions estimates can be compared with MUMBA isoprene measurements during summer of 2012 - 2013.
    Again uncertainties and sensitivities will be examined for various parameters.
    Notably the effects due to smearing and grid resolution will be examined in detail, as these have never been heavily scrutinised for this inversion technique over Australia.
    A detailed comparison against other emissions estimates for Australia (ie: MEGAN and any other inventories) will be done with improvements or limitations highlighted.
    The isoprene emissions product will be analysed further by examination of its simulated VOC products.
  
%   \section{Ozone}
%       
%   \section{HCHO}
%       
%   \section{Isoprene}